\documentclass[aspectratio=169,compress,t,xcolor=table]{beamer}

%! Mandatory packages
\usepackage[utf8]{inputenc}                                %! Character encoding
\usepackage[T1]{fontenc}                                   %! Font encoding
\usepackage[english]{babel}                                %! Language setting
\usepackage{graphicx}

%%%% Frame design
\usetheme{Boadilla}                                        % Design base
\usecolortheme{default}                                    % Color structure base
  \newcommand{\myhfill}[1]{\hskip0pt plus 1filll {\small #1}}
  \newcommand{\sidenote}[1]{\hskip0pt plus 1filll {\fcolorbox{sidenotebg}{sidenotebg}{\small\color{MyStructure} #1}}}

% Colors
\definecolor{MyBackground}{HTML}{FFFAEE}                   % Cream color with dark text works well for dyslexic readers: FFFAEE
\definecolor{sidenotebg}{HTML}{FFE4A8}                   % Cream color with dark text works well for dyslexic readers: FFFAEE
% \definecolor{MyBackground}{HTML}{FFFFFF}                   % White for online slides
\setbeamercolor{background canvas}{bg=MyBackground}
\definecolor{MyStructure}{HTML}{005471}
\setbeamercolor{title}{fg=MyStructure}
\setbeamercolor{frametitle}{fg=MyStructure}
\setbeamercolor{structure}{fg=MyStructure}
\definecolor{emphp}{HTML}{924700}
\newcommand{\emphc}[1]{{\color{emphp}{#1}}}
\definecolor{empho}{HTML}{007211}
\newcommand{\empho}[1]{{\color{empho}{#1}}}
\definecolor{emphh}{HTML}{CF0606}
\newcommand{\emphh}[1]{{\color{emphh}{#1}}}
\definecolor{MyOrange}{HTML}{D17402}
\definecolor{MyPurple}{HTML}{4E008A}
\definecolor{MyOrangeBox}{HTML}{F98A22}
\definecolor{MyPurpleBox}{HTML}{4E008A}

% Fonts
\usepackage{FiraSans}
\beamertemplatenavigationsymbolsempty                      % Disable navigation bar

% Enumerate / itemize design
\setbeamertemplate{enumerate items}[default]
\setbeamertemplate{itemize items}{}
\setbeamertemplate{itemize subitem}[circle]
\setbeamertemplate{section in toc}[sections numbered]

% % Add extra space on the right for Zoom video
% \setbeamersize{text margin right=3.0cm}
% \setlength{\leftmargini}{0pt}
% \setbeamertemplate{background canvas}{\begin{tikzpicture}
% \useasboundingbox (0,0) rectangle (\paperwidth,\paperheight);
% \fill [color=MyStructure] (0.875\paperwidth,0) rectangle (\paperwidth,\paperheight);
% \end{tikzpicture}} 

% Footer
\setbeamertemplate{footline}
{
 \hbox{%
  \begin{beamercolorbox}[wd=.15\paperwidth,ht=2.6ex,dp=1ex,center]{section in foot}%
    \usebeamerfont{section in foot}\insertshortauthor
  \end{beamercolorbox}%
  \begin{beamercolorbox}[wd=.8\paperwidth,ht=2.6ex,dp=1ex,center]{section in foot}%
    \usebeamerfont{section in foot}\insertshorttitle
  \end{beamercolorbox}%
  \begin{beamercolorbox}[wd=.05\paperwidth,ht=2.6ex,dp=1ex,center]{subsection in foot}%
    \usebeamerfont{section in foot} \color{MyStructure}{\insertframenumber}
  \end{beamercolorbox}}%

  \vskip0pt%
}
\setbeamertemplate{frametitle}[default][left]              % Push frame title to left
\setbeamertemplate{navigation symbols}{}                   % Remove navigation symbols

\renewcommand\textbullet{\ensuremath{\bullet}}             % Opress textbullet font warning
\usepackage{appendixnumberbeamer}                          % Restart slide numbering for backup slides (\appendix)

% Roadmap
\usepackage{environ}
\NewEnviron{roadmap}[1][]{%
  \setbeamercolor{background canvas}{bg=MyStructure}
  \setbeamercolor{normal text}{fg=MyBackground}
  \begin{frame}[c]
    \centerline{\huge \color{MyBackground} \textbf{#1}}
  \end{frame}
  \addtocounter{framenumber}{-1}
}

% Text packages
% \usepackage{enumerate}                                     % Custom enumeration lists
\usepackage{footmisc}                                      % Stable footnotes in headings
\usepackage{natbib}                                        % Bibliography
\usepackage{hyperref}                                      % Hyperreferences
    \hypersetup{urlcolor=MyStructure}
\usepackage{multicol}                                      % Multiple columns in text
\usepackage[nointegrals]{wasysym}                          % WASY2 symbols for contradiction \lightning
\usepackage{algorithm}                                     % Pseudocode packages
\usepackage{algpseudocode}
\usepackage{setspace}                                      % Custom line spacing
% Better underline: https://alexwlchan.net/2017/10/latex-underlines/
\usepackage{contour}
\usepackage[normalem]{ulem}
  \renewcommand{\ULdepth}{1.8pt}
  \contourlength{0.8pt}
  \newcommand{\ul}[1]{%
    \uline{\phantom{#1}}%
    \llap{\contour{MyBackground}{#1}}%
  }

% Figure and table packages
\usepackage{float}                                         % Figure floats
\usepackage[position=top,labelformat=empty]{subfig}        % Subfigures
\usepackage{multirow}                                      % Merged rows in tables
\usepackage{dcolumn}                                       % Custom table delimiters
    \newcolumntype{d}[1]{D{.}{.}{#1}}
    \newcommand\hd[1]{\multicolumn{1}{c}{#1}}
\usepackage[labelfont=bf, font=normalsize]{caption}        % Bold captions
    \captionsetup{format=hang}
\usepackage{array}                                         % Reveal table by columns
\usepackage{animate}                                       % Animated figures, GIF-like appearance
\usepackage{media9}                                        % For \mediabutton
\usepackage{booktabs}

% Graphics packages
\usepackage{tikz}                                          % TikZ drawings
  \usetikzlibrary{patterns}
\usepackage{pgfplots}                                      % PGFPlots plots
  \pgfplotsset{compat=1.11}

% Math packages
\usepackage{amsmath}                                       % AMS math package
\usepackage{mathtools}                                     % Extra math tools
\usepackage{amssymb}                                       % Math symbols
\usepackage{amsthm}                                        % Custom theorem environments
\usepackage{units}                                         % Numerical fractions
\usepackage{centernot}                                     % Logical negation in the middle of characters; e.g. not iff
\usepackage{dsfont}                                        % For indicator function: \(\mathds{1}\)

\renewcommand{\qedsymbol}{$\blacksquare$}                  % QED

% Mathematical operators
\DeclareMathOperator{\E}{\mathbb{E}}                       % Expected value
\DeclareMathOperator{\Emax}{\E\!\max}                      % Emax
\DeclareMathOperator{\var}{var}                            % Variance
\DeclareMathOperator{\cov}{cov}                            % Covariance
\DeclareMathOperator{\corr}{corr}                          % Correlation
\DeclareMathOperator{\avar}{avar}                          % Asymptotic variance
\DeclareMathOperator*{\plim}{plim}                         % Probability limit
\DeclareMathOperator{\lag}{lag}                            % Lag operator
\DeclareMathOperator{\rank}{rank}                          % Rank
\DeclareMathOperator{\tr}{tr}                              % Trace
\DeclareMathOperator{\diag}{diag}                          % Diagonal
\DeclareMathOperator{\I}{I}                                % Identity matrix

\providecommand{\abs}[1]{\left\lvert#1\right\rvert}        % Absolute value
\providecommand{\norm}[1]{\left\lVert#1\right\rVert}       % Norm
\providecommand{\ip}[1]{\left\langle#1\right\rangle}       % Inner product
\providecommand{\csp}{\overline{\mathrm{sp}}}              % Closed span
\providecommand{\pto}{\overset{p}{\to}}                    % Convergence in probability
\providecommand{\dto}{\overset{d}{\to}}                    % Convergence in distribution


% File paths
\newcommand{\figpath}{"../../results/graphs/"}
\newcommand{\idfigpath}{"../../results/graphs/identification/"}
\newcommand{\tabpath}{"../../results/tables/"}


\title[]{Empirical Economic Modeling}
\subtitle{Labor}
% % Single author
\author[]{%
  \texorpdfstring{%
    Attila Gyetvai \\ \vspace*{0.5em} \footnotesize Bank of Portugal \& IZA \\ \href{https://attilagyetvai.com}{\color{MyStructure}\ul{attilagyetvai.com}}
  }{Gyetvai}
}
% Multiple authors
% \author[]{%
%   \texorpdfstring{%
%     \begin{columns}
%       \column{.35\textwidth}
%       \centering
%       \normalsize Alastair Abigail \\ \scriptsize Somewhere Good \\ \vspace*{1em}
%       \normalsize Lones Lawry \\ \scriptsize Somewhere Even Better \\
%       \column{.3\textwidth}
%       \centering
%       \normalsize Attila Gyetvai \\ \scriptsize Bank of Portugal, IZA \\ \vspace*{1em}
%       \normalsize John Smith \\ \scriptsize Good Place
%     \end{columns}
%   }
%   {Arcidiacono, Gyetvai, Jardim, Maurel}
% }
\date[]{\footnotesize Spring 2024}


\begin{document}

{
\setbeamertemplate{headline}{}            % Empty header
\setbeamertemplate{footline}{\centerline{\tiny {\color{gray!50!white} The views expressed here are of the author and do not necessarily reflect those of the Bank of Portugal or the Eurosystem.}}\vspace*{10pt}}    % Disclaimer in footer
\begin{frame}
  \titlepage
\end{frame}
}
\addtocounter{framenumber}{-1}

\begin{frame}{Lecture plan}
  \begin{itemize}
    \vfill\item {\color{MyStructure}\textbf{Central research question in labor:}} What determines earnings?
    \vfill\item Theory
    \begin{itemize}
      \addtolength{\baselineskip}{1em}
      \item Labor demand: monopsony power, unintended policy consequences, \ldots
      \item Labor supply: returns to skill, amenities/compensating differentials, luck, \ldots
    \end{itemize}
    \vfill\item Empirics
    \begin{itemize}
      \addtolength{\baselineskip}{1em}
      \item Mincer earnings function
      \item The AKM model
    \end{itemize}
  \end{itemize}
\end{frame}

{
\setbeamercolor{background canvas}{bg=sidenotebg}
\begin{frame}{Journal rankings \myhfill{Sidenote}}
  \begin{itemize}
    \vfill\item Journal quality is a good (not perfect!) proxy for paper quality
    \vfill\item Rankings are subjective---this is mine
  \end{itemize}

  \vfill{\scriptsize
  \begin{center}
    \begin{tabular}{ll}
      \textbf{Top 5} & \textbf{2nd tier general interest} \\
      American Economic Review (AER) & Economic Journal (EJ) \\
      Econometrica (ECMA) & European Economic Review (EER) \\
      Journal of Political Economy (JPE) & Journal of the European Economic Association (JEEA) \\
      Quarterly Journal of Economics (QJE) & Review of Economics and Statistics (REStat) \\
      Review of Economic Studies (REStud) & etc. \\[1em]
      \textbf{Top field---labor} & \textbf{2nd tier field---labor} \\
      American Economic Journal: Applied Economics (AEJ:Applied) & Journal of Human Resources (JHR) \\
      American Economic Journal: Economic Policy (AEJ:Policy) & Labour Economics (LE) \\
      Journal of Labor Economics (JOLE) & etc. \\
    \end{tabular}
  \end{center}
  }
\end{frame}
}

\begin{frame}{Mincer earnings function}
Mincer, Jacob (1958). Investment in Human Capital and Personal Income Distribution. \textit{Journal of Political Economy 66}(4), 281–302.
  \begin{center}
    \includegraphics[height=0.6\textheight]{mincer.jpeg}
  \end{center}
\end{frame}
\addtocounter{framenumber}{-1}

\begin{frame}{Mincer earnings function}
Mincer (1958 JPE). Investment in Human Capital and Personal Income Distribution \\ \phantom{you}
  \begin{center}
    \includegraphics[height=0.6\textheight]{mincer.jpeg}
  \end{center}
\end{frame}
\addtocounter{framenumber}{-1}

\begin{frame}{Mincer earnings function}
Mincer (1958 JPE). Investment in Human Capital and Personal Income Distribution
  \begin{itemize}
    \vfill\item {\color{MyStructure}\textbf{Research question:}} How does schooling vs.\ experience shape earnings?
    \vfill\item Theory:
    \begin{itemize}
      \item Single equation model: for worker \(i\),
      \begin{align*}
         \ln w_{i} = \beta_0 + \beta_1 \text{schooling}_i + \beta_2 \text{experience}_i + \beta_3 \text{experience}_i^2 + u_i
      \end{align*}
      \item No microfoundation (i.e., no behavior captured in a decision model)
    \end{itemize}
    \vfill\item Empirics:
    \begin{itemize}
      \item Estimate model on Census data
      \item \(R^2\) was later found to be low
    \end{itemize}
    \vfill\item Long-lasting impact (c.f.\ Society of Labor Economists, Jacob Mincer Award)
  \end{itemize}
\end{frame}

\begin{frame}{Decomposing wage dispersion}
Abowd, John M., Francis Kramarz and David N. Margolis (1999). High Wage Workers and High Wage Firms. \textit{Econometrica 67}(2), 251-333.
  \begin{center}
    \includegraphics[height=0.6\textheight]{abowd.jpeg} \hspace*{1em}
    \includegraphics[height=0.6\textheight]{kramarz.jpeg} \hspace*{1em}
    \includegraphics[height=0.6\textheight]{margolis.png}
  \end{center}
\end{frame}
\addtocounter{framenumber}{-1}

\begin{frame}{Decomposing wage dispersion}
Abowd, Kramarz and Margolis (1999 ECMA). High Wage Workers and High Wage Firms \\ \phantom{you}
  \begin{center}
    \includegraphics[height=0.6\textheight]{abowd.jpeg} \hspace*{1em}
    \includegraphics[height=0.6\textheight]{kramarz.jpeg} \hspace*{1em}
    \includegraphics[height=0.6\textheight]{margolis.png}
  \end{center}
\end{frame}
\addtocounter{framenumber}{-1}

\begin{frame}{Decomposing wage dispersion}
Abowd, Kramarz and Margolis (1999 ECMA). High Wage Workers and High Wage Firms
  \begin{itemize}
    \vfill\item {\color{MyStructure}\textbf{Research question:}} Are workers or firms the source of wage differences?
    \vfill\item AKM regression: for worker \(i\) at time \(t\) at firm \(j(i,t)\),
    \begin{align*}
       y_{it} = \underbrace{\alpha_{i\phantom{j}}}_{\text{worker FE}} + \underbrace{\phi_{j(i,t)}}_{\text{firm FE}} + x_{it} \beta + u_{it}
    \end{align*}
    \vfill\item Nests the Mincer equation (\(x_{it} \beta\))
    \vfill\item Identified off job-to-job movers (largest connected set)
    \vfill\item Need lots of data! AKM popularized the use of linked employer-employee data
  \end{itemize}
\end{frame}

\begin{frame}{Decomposing wage dispersion (cont'd)}
Abowd, Kramarz and Margolis (1999 ECMA). High Wage Workers and High Wage Firms
  \begin{itemize}
    \vfill\item Variance decomposition: \pause
    \begin{align*}
       \var y_{it} &= \var \alpha_i + \var \phi_{j(i,t)} + \var x_{it} \beta \\
       &+ 2 \cov (\alpha_i, \phi_{j(i,t)}) + 2 \cov (\alpha_i, x_{it} \beta) + 2 \cov(\phi_{j(i,t)}, x_{it} \beta) + \var u_{it}
    \end{align*}
    \vfill\item Commonly found estimates (Lopes de Melo, 2018 JPE):
    \begin{align*}
      \frac{\var \alpha_i}{\var y_{it}} \approx 0.6 \qquad \frac{\var \phi_{j(i,t)}}{\var y_{it}} \approx 0.2 \qquad \frac{\var x_{it}\beta}{\var y_{it}} \approx 0.1
    \end{align*}
    \(\Rightarrow\) differences across workers capture more wage variation than firms
    \vfill\item Covariance terms capture sorting between workers and firms
  \end{itemize}
\end{frame}

\begin{frame}{Where is the theory?}
  \begin{itemize}
    \vfill\item AKM is purposely atheoretical
    \vfill\item Yet it is a very influential tool, used everywhere in economics
    \vfill\item Canned estimation packages (Stata: \texttt{reghdfe} by Correia, 2016)
    \vfill\item Microfoundations?
    \begin{itemize}
      \addtolength{\baselineskip}{0.5em}
      \item Firms: product market power (Wong, 2023 WP AER R\&R)
      \item[] \sidenote{Academic economics publication process}
      \item Workers: half of the labor economics literature \\
      \(\Rightarrow\) we cover one possible explanation next
    \end{itemize}
  \end{itemize}
\end{frame}

\begin{frame}{The role of luck}
McCall, John J. (1970). Economics of Information and Job Search. \textit{Quarterly Journal of Economics 84}(1), 113–126.
  \begin{center}
    \includegraphics[height=0.6\textheight]{mccall.jpg}
  \end{center}
\end{frame}
\addtocounter{framenumber}{-1}

\begin{frame}{The role of luck}
McCall (1970 QJE). Economics of Information and Job Search \\ \phantom{,ou}
  \begin{center}
    \includegraphics[height=0.6\textheight]{mccall.jpg}
  \end{center}
\end{frame}
\addtocounter{framenumber}{-1}

\begin{frame}{The role of luck}
McCall (1970 QJE). Economics of Information and Job Search
  \begin{itemize}
    \vfill\item {\color{MyStructure}\textbf{Research question:}} Why do ex ante identical workers earn different wages?
    \vfill\item Some workers are lucky to get good job offers
    \vfill\item Unemployed get UI benefit \(b\), draw a job offer at rate \(\lambda\) that pays \(w \sim F\)
    \begin{align*}
      (\lambda + \rho) V(b) = b + \lambda \Emax_w \left[ \nicefrac{w}{\rho}, V(b) \right] = b + \lambda \max \left[ \nicefrac{1}{\rho} \int w \, dF(w), V(b) \right]
    \end{align*}
    \vfill\item Solution: reservation wage strategy (accept offer if \(w \geq b\))
  \end{itemize}
\end{frame}

\begin{frame}{The Diamond paradox}
Diamond, Peter A. (1971). A Model of Price Adjustment. \textit{Journal of Economic Theory 3}(2), 156-68.
  \begin{center}
    \includegraphics[height=0.6\textheight]{diamond.jpeg}
  \end{center}
\end{frame}
\addtocounter{framenumber}{-1}

\begin{frame}{The Diamond paradox}
Diamond (1971 JET). A Model of Price Adjustment \\ \phantom{15}
  \begin{center}
    \includegraphics[height=0.6\textheight]{diamond.jpeg}
  \end{center}
\end{frame}
\addtocounter{framenumber}{-1}

\begin{frame}{The Diamond paradox}
Diamond (1971 JET). A Model of Price Adjustment
  \begin{itemize}
    \vfill\item {\color{MyStructure}\textbf{Research question:}} Can wage dispersion arise in equilibrium?
    \vfill\item Wages do not fall from the sky, firms choose them 
    \vfill\item \(\Rightarrow\) wage offers reduce to a point
    \vfill\item \(\Rightarrow\) no equilibrium wage dispersion!
  \end{itemize}
\end{frame}

\begin{frame}{Resolving the Diamond paradox: on-the-job search}
Burdett, Kenneth and Dale Mortensen (1998). Wage Differentials, Employer Size, and Unemployment. \textit{International Economic Review 39}(2), 257-73.
  \begin{center}
    \includegraphics[height=0.6\textheight]{burdett.jpg} \hspace*{1em}
    \includegraphics[height=0.6\textheight]{mortensen.jpeg}
  \end{center}
  \begin{itemize}
    \item[] \phantom{\sidenote{Academic economics publication process 2}}
  \end{itemize}
\end{frame}
\addtocounter{framenumber}{-1}

\begin{frame}{Resolving the Diamond paradox: on-the-job search}
Burdett and Mortensen (1998 IER). Wages, Employer Size, and Unemployment \\ \phantom{you}
  \begin{center}
    \includegraphics[height=0.6\textheight]{burdett.jpg} \hspace*{1em}
    \includegraphics[height=0.6\textheight]{mortensen.jpeg}
  \end{center}
  \begin{itemize}
    \item[] \sidenote{Academic economics publication process 2}
  \end{itemize}
\end{frame}
\addtocounter{framenumber}{-1}

\begin{frame}{Resolving the Diamond paradox: on-the-job search}
Burdett and Mortensen (1998 IER). Wages, Employer Size, and Unemployment
  \begin{itemize}
    \vfill\item {\color{MyStructure}\textbf{Research question:}} Can wage dispersion arise in equilibrium?
    \vfill\item Some workers are lucky to get good job offers \textbf{on the job}
    \vfill\item Unemployed get UI benefit \(b\), draw a job offer at rate \(\lambda_u\) that pays \(w' \sim F\) 
    \vfill\item Employed get \(w\), are fired at rate \(\delta\), draw a job offer at rate \(\lambda_e\) that pays \(w' \sim F\)
    \pause
    \begin{align*}
      (\lambda_u + \rho) V_u(b) &= b + \lambda_u \Emax_{w'} \left[ V_e(w'), V(b) \right] \\[0.5em]
      (\lambda_e + \delta + \rho) V_e(w) &= w + \delta V_u(b) +  \lambda_e \Emax_{w'} \left[ V_e(w'), V_e(w) \right]
    \end{align*}
    \vfill\item This is only the worker side! But framework yields equilibrium wage dispersion
  \end{itemize}
\end{frame}

\begin{frame}{Taking stock}
  \begin{itemize}
    \vfill\item {\color{MyStructure}\textbf{Central research question in labor:}} What determines earnings?
    \vfill\item AKM: unobserved differences between workers are to blame for wage dispersion
    \vfill\item Theory: luck could play a role
    \begin{itemize}
      \item Equilibrium search and matching (2010 Nobel to Diamond--Mortensen--Pissarides)
      \item \textit{Huge} literature (wage bargaining, directed search, amenities, \ldots)
    \end{itemize}
    \vfill\item Empirics: how important a role does luck play?
    \begin{itemize}
      \item Fundamental identification issue: we do not observe rejected offers \(\Rightarrow\) structure!
      \item Calibration: 4--29 percent (Taber and Vejlin, 2020 ECMA)
      \item Computationally intensive, black box
    \end{itemize}
  \end{itemize}
\end{frame}

\begin{frame}{Empirics of job search}
Arcidiacono, Peter S., Attila Gyetvai, Arnaud Maurel and Ekaterina Jardim (2023). Identification and Estimation of Continuous-Time Job Search Models with Preference Shocks. NBER Working Paper 30655, R\&R at the \textit{Review of Economic Studies}.
  \begin{center}
    \includegraphics[height=0.4\textheight]{arcidiacono.jpeg} \hspace*{1em}
    \includegraphics[height=0.4\textheight]{gyetvai.jpg} \hspace*{1em}
    \includegraphics[height=0.4\textheight]{maurel.jpeg} \hspace*{1em}
    \includegraphics[height=0.4\textheight]{jardim.jpeg}
  \end{center}
\end{frame}
\addtocounter{framenumber}{-1}

\begin{frame}{Empirics of job search}
Arcidiacono, Gyetvai, Maurel and Jardim (2023 NBER WP). CCP Search \\ \phantom{Identification and Estimation of Continuous-Time Job Search Models with Preference Shocks} \\ \phantom{NBER Working Paper 30655, R\&R at the \textit{Review of Economic Studies}.}
  \begin{center}
    \includegraphics[height=0.4\textheight]{arcidiacono.jpeg} \hspace*{1em}
    \includegraphics[height=0.4\textheight]{gyetvai.jpg} \hspace*{1em}
    \includegraphics[height=0.4\textheight]{maurel.jpeg} \hspace*{1em}
    \includegraphics[height=0.4\textheight]{jardim.jpeg}
  \end{center}
\end{frame}
\addtocounter{framenumber}{-1}

\begin{frame}{Empirics of job search}
Arcidiacono, Gyetvai, Maurel and Jardim (2023 NBER WP). CCP Search 
  \begin{itemize}
    \vfill\item {\color{MyStructure}\textbf{Research question:}} How to identify the structural parameters of search models?
    \vfill\item Key idea: draw wage \(w' \sim F\) and preference shock \({\color{red}\varepsilon} \sim \text{Logistic}\)
    \begin{align*}
      (\lambda_u + \rho) V_u(b) &= b + \lambda_u \Emax_{w', {\color{red}\varepsilon}} \left[ V_e(w') + {\color{red}\varepsilon}, V(b) \right] \\[0.5em]
      (\lambda_e + \delta + \rho) V_e(w) &= w + \delta V_u(b) +  \lambda_e \Emax_{w', {\color{red}\varepsilon}} \left[ V_e(w') + {\color{red}\varepsilon}, V_e(w) \right]
    \end{align*}
    \vfill\item Structural parameters are expressed in terms of conditional choice probabilities:
    \begin{align*}
      p_{ww'} = \frac{\exp(V_e(w'))}{\exp(V_e(w)) + \exp(V_e(w'))}
    \end{align*}
  \end{itemize}
\end{frame}

\begin{frame}{Empirics of job search (cont'd)}
Arcidiacono, Gyetvai, Maurel and Jardim (2023 NBER WP). CCP Search 
  \begin{itemize}
    \vfill\item {\color{MyStructure}\textbf{Research question:}} How to identify the structural parameters of search models?
    \vfill\item Conditional choice probability (CCP) methods (IO literature, c.f.\ Rust, 1987)
    \vfill\item Key difference: CCPs are not observed \(\Rightarrow\) we use the logit structure
    \vfill\item Estimation is much easier than calibration before
    \vfill\item Gyetvai (2024 WP): application to occupational mobility
    \vfill\item[] \sidenote{Academic economics job market}
  \end{itemize}
\end{frame}

\begin{frame}{Frontiers of labor}
  \begin{itemize}
    \vfill\item {\color{MyStructure}\textbf{Central research question in labor:}} What determines earnings?
    \begin{itemize}
      \item Luck: we got to one point on the frontier
      \item \ldots
    \end{itemize}
    \vfill\item Other central questions: 
    \begin{itemize}
      \item What determines employment status? \sidenote{Extensive vs.\ intensive margin}
      \item What determines firms' organization structure?
      \item \ldots
    \end{itemize}
    \vfill\item Things we do not cover: 
    \begin{itemize}
      \item Human capital formation/education
      \item Scarring effect of unemployment, motherhood, health shocks, \ldots
      \item Household decisions, matching on the marriage market
      \item Networks
      \item Minimum wage debate
      \item Labor market power
      \item \ldots
    \end{itemize}
  \end{itemize}
\end{frame}


\end{document}

