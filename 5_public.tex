\documentclass[aspectratio=169,compress,t,xcolor=table]{beamer}

%! Mandatory packages
\usepackage[utf8]{inputenc}                                %! Character encoding
\usepackage[T1]{fontenc}                                   %! Font encoding
\usepackage[english]{babel}                                %! Language setting
\usepackage{graphicx}

%%%% Frame design
\usetheme{Boadilla}                                        % Design base
\usecolortheme{default}                                    % Color structure base
  \newcommand{\myhfill}[1]{\hskip0pt plus 1filll {\small #1}}
  \newcommand{\sidenote}[1]{\hskip0pt plus 1filll {\fcolorbox{sidenotebg}{sidenotebg}{\small\color{MyStructure} #1}}}

% Colors
\definecolor{MyBackground}{HTML}{FFFAEE}                   % Cream color with dark text works well for dyslexic readers: FFFAEE
\definecolor{sidenotebg}{HTML}{FFE4A8}                   % Cream color with dark text works well for dyslexic readers: FFFAEE
% \definecolor{MyBackground}{HTML}{FFFFFF}                   % White for online slides
\setbeamercolor{background canvas}{bg=MyBackground}
\definecolor{MyStructure}{HTML}{005471}
\setbeamercolor{title}{fg=MyStructure}
\setbeamercolor{frametitle}{fg=MyStructure}
\setbeamercolor{structure}{fg=MyStructure}
\definecolor{emphp}{HTML}{924700}
\newcommand{\emphc}[1]{{\color{emphp}{#1}}}
\definecolor{empho}{HTML}{007211}
\newcommand{\empho}[1]{{\color{empho}{#1}}}
\definecolor{emphh}{HTML}{CF0606}
\newcommand{\emphh}[1]{{\color{emphh}{#1}}}
\definecolor{MyOrange}{HTML}{D17402}
\definecolor{MyPurple}{HTML}{4E008A}
\definecolor{MyOrangeBox}{HTML}{F98A22}
\definecolor{MyPurpleBox}{HTML}{4E008A}

% Fonts
\usepackage{FiraSans}
\beamertemplatenavigationsymbolsempty                      % Disable navigation bar

% Enumerate / itemize design
\setbeamertemplate{enumerate items}[default]
\setbeamertemplate{itemize items}{}
\setbeamertemplate{itemize subitem}[circle]
\setbeamertemplate{section in toc}[sections numbered]

% % Add extra space on the right for Zoom video
% \setbeamersize{text margin right=3.0cm}
% \setlength{\leftmargini}{0pt}
% \setbeamertemplate{background canvas}{\begin{tikzpicture}
% \useasboundingbox (0,0) rectangle (\paperwidth,\paperheight);
% \fill [color=MyStructure] (0.875\paperwidth,0) rectangle (\paperwidth,\paperheight);
% \end{tikzpicture}} 

% Footer
\setbeamertemplate{footline}
{
 \hbox{%
  \begin{beamercolorbox}[wd=.15\paperwidth,ht=2.6ex,dp=1ex,center]{section in foot}%
    \usebeamerfont{section in foot}\insertshortauthor
  \end{beamercolorbox}%
  \begin{beamercolorbox}[wd=.8\paperwidth,ht=2.6ex,dp=1ex,center]{section in foot}%
    \usebeamerfont{section in foot}\insertshorttitle
  \end{beamercolorbox}%
  \begin{beamercolorbox}[wd=.05\paperwidth,ht=2.6ex,dp=1ex,center]{subsection in foot}%
    \usebeamerfont{section in foot} \color{MyStructure}{\insertframenumber}
  \end{beamercolorbox}}%

  \vskip0pt%
}
\setbeamertemplate{frametitle}[default][left]              % Push frame title to left
\setbeamertemplate{navigation symbols}{}                   % Remove navigation symbols

\renewcommand\textbullet{\ensuremath{\bullet}}             % Opress textbullet font warning
\usepackage{appendixnumberbeamer}                          % Restart slide numbering for backup slides (\appendix)

% Roadmap
\usepackage{environ}
\NewEnviron{roadmap}[1][]{%
  \setbeamercolor{background canvas}{bg=MyStructure}
  \setbeamercolor{normal text}{fg=MyBackground}
  \begin{frame}[c]
    \centerline{\huge \color{MyBackground} \textbf{#1}}
  \end{frame}
  \addtocounter{framenumber}{-1}
}

% Text packages
% \usepackage{enumerate}                                     % Custom enumeration lists
\usepackage{footmisc}                                      % Stable footnotes in headings
\usepackage{natbib}                                        % Bibliography
\usepackage{hyperref}                                      % Hyperreferences
    \hypersetup{urlcolor=MyStructure}
\usepackage{multicol}                                      % Multiple columns in text
\usepackage[nointegrals]{wasysym}                          % WASY2 symbols for contradiction \lightning
\usepackage{algorithm}                                     % Pseudocode packages
\usepackage{algpseudocode}
\usepackage{setspace}                                      % Custom line spacing
% Better underline: https://alexwlchan.net/2017/10/latex-underlines/
\usepackage{contour}
\usepackage[normalem]{ulem}
  \renewcommand{\ULdepth}{1.8pt}
  \contourlength{0.8pt}
  \newcommand{\ul}[1]{%
    \uline{\phantom{#1}}%
    \llap{\contour{MyBackground}{#1}}%
  }
  \newcommand{\uly}[1]{%
    \uline{\phantom{#1}}%
    \llap{\contour{sidenotebg}{#1}}%
  }

% Figure and table packages
\usepackage{float}                                         % Figure floats
\usepackage[position=top,labelformat=empty]{subfig}        % Subfigures
\usepackage{multirow}                                      % Merged rows in tables
\usepackage{dcolumn}                                       % Custom table delimiters
    \newcolumntype{d}[1]{D{.}{.}{#1}}
    \newcommand\hd[1]{\multicolumn{1}{c}{#1}}
\usepackage[labelfont=bf, font=normalsize]{caption}        % Bold captions
    \captionsetup{format=hang}
\usepackage{array}                                         % Reveal table by columns
\usepackage{animate}                                       % Animated figures, GIF-like appearance
\usepackage{media9}                                        % For \mediabutton
\usepackage{booktabs}

% Graphics packages
\usepackage{tikz}                                          % TikZ drawings
  \usetikzlibrary{patterns}
\usepackage{pgfplots}                                      % PGFPlots plots
  \pgfplotsset{compat=1.11}

% Math packages
\usepackage{amsmath}                                       % AMS math package
\usepackage{mathtools}                                     % Extra math tools
\usepackage{amssymb}                                       % Math symbols
\usepackage{amsthm}                                        % Custom theorem environments
\usepackage{units}                                         % Numerical fractions
\usepackage{centernot}                                     % Logical negation in the middle of characters; e.g. not iff
\usepackage{dsfont}                                        % For indicator function: \(\mathds{1}\)

\renewcommand{\qedsymbol}{$\blacksquare$}                  % QED

% Mathematical operators
\DeclareMathOperator{\E}{\mathbb{E}}                       % Expected value
\DeclareMathOperator{\Emax}{\E\!\max}                      % Emax
\DeclareMathOperator{\var}{var}                            % Variance
\DeclareMathOperator{\cov}{cov}                            % Covariance
\DeclareMathOperator{\corr}{corr}                          % Correlation
\DeclareMathOperator{\avar}{avar}                          % Asymptotic variance
\DeclareMathOperator*{\plim}{plim}                         % Probability limit
\DeclareMathOperator{\lag}{lag}                            % Lag operator
\DeclareMathOperator{\rank}{rank}                          % Rank
\DeclareMathOperator{\tr}{tr}                              % Trace
\DeclareMathOperator{\diag}{diag}                          % Diagonal
\DeclareMathOperator{\I}{I}                                % Identity matrix

\providecommand{\abs}[1]{\left\lvert#1\right\rvert}        % Absolute value
\providecommand{\norm}[1]{\left\lVert#1\right\rVert}       % Norm
\providecommand{\ip}[1]{\left\langle#1\right\rangle}       % Inner product
\providecommand{\csp}{\overline{\mathrm{sp}}}              % Closed span
\providecommand{\pto}{\overset{p}{\to}}                    % Convergence in probability
\providecommand{\dto}{\overset{d}{\to}}                    % Convergence in distribution


% File paths
\newcommand{\figpath}{"../../results/graphs/"}
\newcommand{\idfigpath}{"../../results/graphs/identification/"}
\newcommand{\tabpath}{"../../results/tables/"}


\title[]{Empirical Economic Modeling}
\subtitle{Public}
% % Single author
\author[]{%
  \texorpdfstring{%
    Attila Gyetvai \\ \vspace*{0.5em} \footnotesize Bank of Portugal \& IZA \\ \href{https://attilagyetvai.com}{\color{MyStructure}\ul{attilagyetvai.com}}
  }{Gyetvai}
}
% Multiple authors
% \author[]{%
%   \texorpdfstring{%
%     \begin{columns}
%       \column{.35\textwidth}
%       \centering
%       \normalsize Alastair Abigail \\ \scriptsize Somewhere Good \\ \vspace*{1em}
%       \normalsize Lones Lawry \\ \scriptsize Somewhere Even Better \\
%       \column{.3\textwidth}
%       \centering
%       \normalsize Attila Gyetvai \\ \scriptsize Bank of Portugal, IZA \\ \vspace*{1em}
%       \normalsize John Smith \\ \scriptsize Good Place
%     \end{columns}
%   }
%   {Arcidiacono, Gyetvai, Jardim, Maurel}
% }
\date[]{\footnotesize Spring 2024}


\begin{document}

{
\setbeamertemplate{headline}{}            % Empty header
\setbeamertemplate{footline}{\centerline{\tiny {\color{gray!50!white} The views expressed here are of the author and do not necessarily reflect those of the Bank of Portugal or the Eurosystem.}}\vspace*{10pt}}    % Disclaimer in footer
\begin{frame}
  \titlepage
\end{frame}
}
\addtocounter{framenumber}{-1}

\begin{frame}{Lecture plan}
  \begin{itemize}
    \vfill\item {\color{MyStructure}\textbf{Central research questions in public:}} How to design tax systems?
    \begin{enumerate}
      \addtolength{\baselineskip}{1em}
      \item How do various forms of taxation compare?
      \item How does behavior impact taxation?
    \end{enumerate}
  \end{itemize}
\end{frame}

\begin{frame}{}
Guvenen, Fatih, Gueorgui Kambourov, Burhan Kuruscu, Sergio Ocampo and Daphne Chen (2023). Use It or Lose It: Efficiency and Redistributional Effects of Wealth Taxation. \textit{Quarterly Journal of Economics 138}(2), 835-894.
  \begin{center}
    \includegraphics[height=0.3\textheight]{guvenen.jpeg} \hspace*{1em}
    \includegraphics[height=0.3\textheight]{kambourov.jpg} \hspace*{1em}
    \includegraphics[height=0.3\textheight]{kuruscu.jpeg} \hspace*{1em}
    \includegraphics[height=0.3\textheight]{ocampo.jpg} \hspace*{1em}
    \includegraphics[height=0.3\textheight]{chen.png}
  \end{center}
\end{frame}
\addtocounter{framenumber}{-1}

\begin{frame}{}
Guvenen, Kambourov, Kuruscu, Ocampo and Chen (2023 QJE). Use It or Lose It \\ \phantom{(Efficiency)} \\ \phantom{Journal of you,}
  \begin{center}
    \includegraphics[height=0.3\textheight]{guvenen.jpeg} \hspace*{1em}
    \includegraphics[height=0.3\textheight]{kambourov.jpg} \hspace*{1em}
    \includegraphics[height=0.3\textheight]{kuruscu.jpeg} \hspace*{1em}
    \includegraphics[height=0.3\textheight]{ocampo.jpg} \hspace*{1em}
    \includegraphics[height=0.3\textheight]{chen.png}
  \end{center}
\end{frame}
\addtocounter{framenumber}{-1}

\begin{frame}{}
Guvenen, Kambourov, Kuruscu, Ocampo and Chen (2023 QJE). Use It or Lose It 
  \begin{itemize}
    \item {\color{MyStructure}\textbf{Research question:}} How does wealth taxation compare to capital taxation?
    \vfill\item Example: Fredo is bad at business, Michael is good. Government needs to raise \$5M 
    \begin{center}
      \includegraphics[width=0.6\textwidth]{gkkoc_tab1.png}
    \end{center}
    \vfill\item \(\Rightarrow\) Under wealth taxation, you must use your wealth or you lose it
    \vfill\item If Fredo would be as good as Michael, form of taxation would not matter!
    \vfill\item \(\leadsto\) Taxing capital vs.\ wealth is the same with homogeneous rate of return \sidenote{HANK}
    \vfill\item \ldots but we have empirical evidence that \(r\) is heterogeneous
  \end{itemize}
\end{frame}

\begin{frame}{}
Guvenen, Kambourov, Kuruscu, Ocampo and Chen (2023 QJE). Use It or Lose It 
  \begin{itemize}
    \item {\color{MyStructure}\textbf{Research question:}} How does wealth taxation compare to capital taxation?
    \vfill\item \textit{Entrepreneurial productivity}
    \vfill\item Everyone is born with ability \(\bar{z}_i\): \(\log \bar{z}_i = \rho_z \log \bar{z}_i^{\text{parent}} + \varepsilon_{\bar{z}_i}\), \(\varepsilon_{\bar{z}_i} \sim N(0,\sigma_{\bar{z}_i}^2)\)
    \vfill\item \quad \(\Rightarrow\) Some high-ability children will start with low wealth and vice versa
    \vfill\item \qquad \(\Rightarrow\) Capital misallocation!
    \vfill\item Productivity evolves: at age \(h\),
    \begin{align*}
      z_{ih} = \begin{cases}
        \bar{z}_i^{\lambda} & \text{if high} \\
        \bar{z}_i & \text{if low} \\
        0  & \text{if out}
      \end{cases}
      \qquad \text{with} \enskip
      \Pi = \begin{bmatrix}
        1-p_1-p_2 & p_1 & p_2 \\
        0 & 1-p_2 & p_2 \\
        0 & 0 & 1
      \end{bmatrix}
    \end{align*}
  \end{itemize}
\end{frame}

\begin{frame}{}
Guvenen, Kambourov, Kuruscu, Ocampo and Chen (2023 QJE). Use It or Lose It 
  \begin{itemize}
    \item {\color{MyStructure}\textbf{Research question:}} How does wealth taxation compare to capital taxation?
    \vfill\item \textit{Labor productivity}
    \vfill\item Everyone is born with ability \(\kappa_i\): \(\kappa_i = \rho_{\kappa} \kappa_i^{\text{parent}} + \varepsilon_{\kappa_i}\), \(\varepsilon_{\kappa_i} \sim N(0,\sigma_{\kappa_i}^2)\)
    \vfill\item Productivity evolves: at age \(h\),
    \begin{align*}
      \log w_{ih} = \kappa_i + g(h) + e_{ih}\text{,} \qquad e_{ih} \sim AR(1)
    \end{align*}
    \vfill\item \quad \(\Rightarrow\) Aggregate labor supply:
    \begin{align*}
      L = \int w_{i,h(i)} \ell_{i,h(i)} \, di
    \end{align*}
  \end{itemize}
\end{frame}

\begin{frame}{}
Guvenen, Kambourov, Kuruscu, Ocampo and Chen (2023 QJE). Use It or Lose It 
  \begin{itemize}
    \item {\color{MyStructure}\textbf{Research question:}} How does wealth taxation compare to capital taxation?
    \vfill\item \textit{Production}
    \vfill\item Entrepreneurs produce intermediate good \(x_{ih} = z_{ih} k_{ih}\)
    \vfill\item Final good is produced by combining intermediate inputs and labor:
    \begin{align*}
      Y = Q^{\alpha} L^{1-\alpha} = \left( \int x_{i,h(i)}^{\mu} \, di \right)^{\frac{\alpha}{\mu}} L^{1-\alpha}
    \end{align*}
    \vfill\item \textit{Consumption, labor, bequests}
    \vfill\item Individuals maximize expected lifetime utility:
    \begin{align*}
      \max \, \E \left[ \sum_{h=1}^H \beta^{h-1} (\phi_h u(c_h, 1-\ell_h) + (1-\phi_h) v(b)) \right]
    \end{align*}
  \end{itemize}
\end{frame}

\begin{frame}{}
Guvenen, Kambourov, Kuruscu, Ocampo and Chen (2023 QJE). Use It or Lose It 
  \begin{itemize}
    \item {\color{MyStructure}\textbf{Research question:}} How does wealth taxation compare to capital taxation?
    \vfill\item \textit{Financial markets}
    \item Assets \(a_{ih}\) can be borrowed subject to some borrowing constraint
    \vfill\item \textit{Government}
    \item Capital taxation: taxes on capital income, labor, consumption, bequests
    \item Wealth taxation: taxes on assets, labor, consumption, bequests
    \item (Also pensions)
  \end{itemize}
\end{frame}

\begin{frame}{}
Guvenen, Kambourov, Kuruscu, Ocampo and Chen (2023 QJE). Use It or Lose It 
  \begin{itemize}
    \item {\color{MyStructure}\textbf{Research question:}} How does wealth taxation compare to capital taxation?
    \vfill\item \textit{Empirics}
    \item Calibrate model on benchmark US economy with capital taxation
    \item Compare aggregates under counterfactual wealth taxation
    \begin{center}
      \includegraphics[width=0.6\textwidth]{gkkoc_tab5.png}
    \end{center}
    \item RN vs.\ BB: due to pensions
    \item Lots more in paper: welfare, optimal taxation, progressive taxes, transition path, \ldots
  \end{itemize}
\end{frame}


\end{document}

