\documentclass[aspectratio=169,compress,t,xcolor=table]{beamer}

%! Mandatory packages
\usepackage[utf8]{inputenc}                                %! Character encoding
\usepackage[T1]{fontenc}                                   %! Font encoding
\usepackage[english]{babel}                                %! Language setting
\usepackage{graphicx}

%%%% Frame design
\usetheme{Boadilla}                                        % Design base
\usecolortheme{default}                                    % Color structure base
  \newcommand{\myhfill}[1]{\hskip0pt plus 1filll {\small #1}}
  \newcommand{\sidenote}[1]{\hskip0pt plus 1filll {\fcolorbox{sidenotebg}{sidenotebg}{\small\color{MyStructure} #1}}}

% Colors
\definecolor{MyBackground}{HTML}{FFFAEE}                   % Cream color with dark text works well for dyslexic readers: FFFAEE
\definecolor{sidenotebg}{HTML}{FFE4A8}                   % Cream color with dark text works well for dyslexic readers: FFFAEE
% \definecolor{MyBackground}{HTML}{FFFFFF}                   % White for online slides
\setbeamercolor{background canvas}{bg=MyBackground}
\definecolor{MyStructure}{HTML}{005471}
\setbeamercolor{title}{fg=MyStructure}
\setbeamercolor{frametitle}{fg=MyStructure}
\setbeamercolor{structure}{fg=MyStructure}
\definecolor{emphp}{HTML}{924700}
\newcommand{\emphc}[1]{{\color{emphp}{#1}}}
\definecolor{empho}{HTML}{007211}
\newcommand{\empho}[1]{{\color{empho}{#1}}}
\definecolor{emphh}{HTML}{CF0606}
\newcommand{\emphh}[1]{{\color{emphh}{#1}}}
\definecolor{MyOrange}{HTML}{D17402}
\definecolor{MyPurple}{HTML}{4E008A}
\definecolor{MyOrangeBox}{HTML}{F98A22}
\definecolor{MyPurpleBox}{HTML}{4E008A}

% Fonts
\usepackage{FiraSans}
\beamertemplatenavigationsymbolsempty                      % Disable navigation bar

% Enumerate / itemize design
\setbeamertemplate{enumerate items}[default]
\setbeamertemplate{itemize items}{}
\setbeamertemplate{itemize subitem}[circle]
\setbeamertemplate{section in toc}[sections numbered]

% % Add extra space on the right for Zoom video
% \setbeamersize{text margin right=3.0cm}
% \setlength{\leftmargini}{0pt}
% \setbeamertemplate{background canvas}{\begin{tikzpicture}
% \useasboundingbox (0,0) rectangle (\paperwidth,\paperheight);
% \fill [color=MyStructure] (0.875\paperwidth,0) rectangle (\paperwidth,\paperheight);
% \end{tikzpicture}} 

% Footer
\setbeamertemplate{footline}
{
 \hbox{%
  \begin{beamercolorbox}[wd=.15\paperwidth,ht=2.6ex,dp=1ex,center]{section in foot}%
    \usebeamerfont{section in foot}\insertshortauthor
  \end{beamercolorbox}%
  \begin{beamercolorbox}[wd=.8\paperwidth,ht=2.6ex,dp=1ex,center]{section in foot}%
    \usebeamerfont{section in foot}\insertshorttitle
  \end{beamercolorbox}%
  \begin{beamercolorbox}[wd=.05\paperwidth,ht=2.6ex,dp=1ex,center]{subsection in foot}%
    \usebeamerfont{section in foot} \color{MyStructure}{\insertframenumber}
  \end{beamercolorbox}}%

  \vskip0pt%
}
\setbeamertemplate{frametitle}[default][left]              % Push frame title to left
\setbeamertemplate{navigation symbols}{}                   % Remove navigation symbols

\renewcommand\textbullet{\ensuremath{\bullet}}             % Opress textbullet font warning
\usepackage{appendixnumberbeamer}                          % Restart slide numbering for backup slides (\appendix)

% Roadmap
\usepackage{environ}
\NewEnviron{roadmap}[1][]{%
  \setbeamercolor{background canvas}{bg=MyStructure}
  \setbeamercolor{normal text}{fg=MyBackground}
  \begin{frame}[c]
    \centerline{\huge \color{MyBackground} \textbf{#1}}
  \end{frame}
  \addtocounter{framenumber}{-1}
}

% Text packages
% \usepackage{enumerate}                                     % Custom enumeration lists
\usepackage{footmisc}                                      % Stable footnotes in headings
\usepackage{natbib}                                        % Bibliography
\usepackage{hyperref}                                      % Hyperreferences
    \hypersetup{urlcolor=MyStructure}
\usepackage{multicol}                                      % Multiple columns in text
\usepackage[nointegrals]{wasysym}                          % WASY2 symbols for contradiction \lightning
\usepackage{algorithm}                                     % Pseudocode packages
\usepackage{algpseudocode}
\usepackage{setspace}                                      % Custom line spacing
% Better underline: https://alexwlchan.net/2017/10/latex-underlines/
\usepackage{contour}
\usepackage[normalem]{ulem}
  \renewcommand{\ULdepth}{1.8pt}
  \contourlength{0.8pt}
  \newcommand{\ul}[1]{%
    \uline{\phantom{#1}}%
    \llap{\contour{MyBackground}{#1}}%
  }

% Figure and table packages
\usepackage{float}                                         % Figure floats
\usepackage[position=top,labelformat=empty]{subfig}        % Subfigures
\usepackage{multirow}                                      % Merged rows in tables
\usepackage{dcolumn}                                       % Custom table delimiters
    \newcolumntype{d}[1]{D{.}{.}{#1}}
    \newcommand\hd[1]{\multicolumn{1}{c}{#1}}
\usepackage[labelfont=bf, font=normalsize]{caption}        % Bold captions
    \captionsetup{format=hang}
\usepackage{array}                                         % Reveal table by columns
\usepackage{animate}                                       % Animated figures, GIF-like appearance
\usepackage{media9}                                        % For \mediabutton
\usepackage{booktabs}

% Graphics packages
\usepackage{tikz}                                          % TikZ drawings
  \usetikzlibrary{patterns}
\usepackage{pgfplots}                                      % PGFPlots plots
  \pgfplotsset{compat=1.11}

% Math packages
\usepackage{amsmath}                                       % AMS math package
\usepackage{mathtools}                                     % Extra math tools
\usepackage{amssymb}                                       % Math symbols
\usepackage{amsthm}                                        % Custom theorem environments
\usepackage{units}                                         % Numerical fractions
\usepackage{centernot}                                     % Logical negation in the middle of characters; e.g. not iff
\usepackage{dsfont}                                        % For indicator function: \(\mathds{1}\)

\renewcommand{\qedsymbol}{$\blacksquare$}                  % QED

% Mathematical operators
\DeclareMathOperator{\E}{\mathbb{E}}                       % Expected value
\DeclareMathOperator{\Emax}{\E\!\max}                      % Emax
\DeclareMathOperator{\var}{var}                            % Variance
\DeclareMathOperator{\cov}{cov}                            % Covariance
\DeclareMathOperator{\corr}{corr}                          % Correlation
\DeclareMathOperator{\avar}{avar}                          % Asymptotic variance
\DeclareMathOperator*{\plim}{plim}                         % Probability limit
\DeclareMathOperator{\lag}{lag}                            % Lag operator
\DeclareMathOperator{\rank}{rank}                          % Rank
\DeclareMathOperator{\tr}{tr}                              % Trace
\DeclareMathOperator{\diag}{diag}                          % Diagonal
\DeclareMathOperator{\I}{I}                                % Identity matrix

\providecommand{\abs}[1]{\left\lvert#1\right\rvert}        % Absolute value
\providecommand{\norm}[1]{\left\lVert#1\right\rVert}       % Norm
\providecommand{\ip}[1]{\left\langle#1\right\rangle}       % Inner product
\providecommand{\csp}{\overline{\mathrm{sp}}}              % Closed span
\providecommand{\pto}{\overset{p}{\to}}                    % Convergence in probability
\providecommand{\dto}{\overset{d}{\to}}                    % Convergence in distribution


% File paths
\newcommand{\figpath}{"../../results/graphs/"}
\newcommand{\idfigpath}{"../../results/graphs/identification/"}
\newcommand{\tabpath}{"../../results/tables/"}


\title[]{Empirical Economic Modeling}
\subtitle{Industrial organization}
% % Single author
\author[]{%
  \texorpdfstring{%
    Attila Gyetvai \\ \vspace*{0.5em} \footnotesize Bank of Portugal \& IZA \\ \href{https://attilagyetvai.com}{\color{MyStructure}\ul{attilagyetvai.com}}
  }{Gyetvai}
}
% Multiple authors
% \author[]{%
%   \texorpdfstring{%
%     \begin{columns}
%       \column{.35\textwidth}
%       \centering
%       \normalsize Alastair Abigail \\ \scriptsize Somewhere Good \\ \vspace*{1em}
%       \normalsize Lones Lawry \\ \scriptsize Somewhere Even Better \\
%       \column{.3\textwidth}
%       \centering
%       \normalsize Attila Gyetvai \\ \scriptsize Bank of Portugal, IZA \\ \vspace*{1em}
%       \normalsize John Smith \\ \scriptsize Good Place
%     \end{columns}
%   }
%   {Arcidiacono, Gyetvai, Jardim, Maurel}
% }
\date[]{\footnotesize Spring 2024}


\begin{document}

{
\setbeamertemplate{headline}{}            % Empty header
\setbeamertemplate{footline}{\centerline{\tiny {\color{gray!50!white} The views expressed here are of the author and do not necessarily reflect those of the Bank of Portugal or the Eurosystem.}}\vspace*{10pt}}    % Disclaimer in footer
\begin{frame}
  \titlepage
\end{frame}
}
\addtocounter{framenumber}{-1}

\begin{frame}{Lecture plan}
  \begin{itemize}
    \vfill\item {\color{MyStructure}\textbf{Central research question in IO:}} How do firms produce?
    \begin{itemize}
      \addtolength{\baselineskip}{1em}
      \item What is the level of returns to scale?
      \item How do input coefficients on capital and labor change over time?
      \item How does adoption of a new technology affect production?
      \item How much heterogeneity is there in measured productivity across firms?
      \item What explains such productivity differences?
      \item How does the allocation of firm inputs relate to productivity?
    \end{itemize}
    \vspace*{1em}
    {\scriptsize Source: Dan Ackerberg's 2017 AEA mini course (\href{https://www.aeaweb.org/content/file?id=3015}{https://www.aeaweb.org/content/file?id=3015})}
  \end{itemize}
\end{frame}

\begin{frame}{}
Olley, G.\ Steve and Ariel Pakes (1996). The Dynamics of Productivity in the Telecommunications Equipment Industry. \textit{Econometrica 64}(6), 1263-1297.
  \begin{center}
    \includegraphics[height=0.4\textheight]{olley.jpeg} \hspace*{1em}
    \includegraphics[height=0.4\textheight]{pakes.jpeg}
  \end{center}
\end{frame}
\addtocounter{framenumber}{-1}

\begin{frame}{}
Olley  and Pakes (1996 ECMA). Dynamics of Productivity in Telcom \\ \phantom{you}
  \begin{center}
    \includegraphics[height=0.4\textheight]{olley.jpeg} \hspace*{1em}
    \includegraphics[height=0.4\textheight]{pakes.jpeg}
  \end{center}
\end{frame}
\addtocounter{framenumber}{-1}

\begin{frame}{}
Olley  and Pakes (1996 ECMA). Dynamics of Productivity in Telcom
  \begin{itemize}
    \item {\color{MyStructure}\textbf{Research question:}} How to estimate production functions?
    \vfill\item (Log) value added production function with labor and capital inputs:
    \begin{align*}
      va_{jt} = \beta_0 + \beta_l l_{jt} + \beta_k k_{jt} + \omega_{jt} + \varepsilon_{jt}
    \end{align*}
    \vfill\item \(\omega_{jt}\): Markovian firm productivity (unobserved by economist); \(\omega_{jt} = g(\omega_{jt-1}) + \xi_{jt}\) \\[0.5em]
    \vfill\item \(\varepsilon_{jt}\): transitory shock
    \vfill\item \sidenote{What if we estimate this regression with OLS? \pause Omitted variable bias!}
    \vfill\item Timing: \(k_{jt}\) is predetermined (chosen before \(t\)) \quad \(\Rightarrow \E [\xi_{jt} \,|\, k_{jt}] = 0\) \quad \(\Rightarrow \E [\xi_{jt} k_{jt}] = 0\)
  \end{itemize}
\end{frame}

\begin{frame}{}
Olley  and Pakes (1996 ECMA). Dynamics of Productivity in Telcom
  \begin{itemize}
    \item {\color{MyStructure}\textbf{Research question:}} How to estimate production functions?
    \vfill\item {\color{MyStructure}\textit{Million dollar idea:}} use investment to back out productivity
    \begin{align*}
      i_{jt} &= \tilde{i}_t(k_{jt}, \omega_{jt}) \quad \Rightarrow \omega_{jt} = \tilde{i}^{-1}_t(k_{jt}, i_{jt}) \quad \text{if \(\tilde{i}_t(\cdot)\) is monotonic in \(\omega_{jt}\)} \\
      \Rightarrow va_{jt} &= \beta_0 + \beta_l l_{jt} + \underbrace{\beta_k k_{jt} + \tilde{i}^{-1}_t(k_{jt}, i_{jt})}_{\doteq \gamma_t(k_{jt}, i_{jt})} + \varepsilon_{jt}
    \end{align*}
    \vfill\item[1.] Regress \(va_{jt}\) on \(l_{jt}\), \(f(k_{jt},i_{jt})\) \quad \(\Rightarrow \text{get } \hat{\beta}_l\), \(\hat{\gamma}_t(\cdot)\)
    \vfill\item[2.] Get \(\hat{\beta}_k\) from moment condition \(\E [\xi_{jt} k_{jt}] = 0\)
    \begin{enumerate}[a.]
      \item Given a guess for \(\beta_k\), write \(\hat{\omega}_{jt} (\beta_k) = \hat{\gamma}_t (k_{jt}, i_{jt}) - \beta_k k_{jt}\)
      \item Regress \(\hat{\omega}_{jt} (\beta_k)\) nonparametrically on \(\hat{\omega}_{jt-1} (\beta_k)\) \quad \(\Rightarrow \text{get } \hat{\xi}_{jt} (\beta_k)\)
      \item Evaluate the sample analogue of the moment condition: \(\frac{1}{NT} \sum_{j, t} \hat{\xi}_{jt} (\beta_k) \, k_{jt}\) \\[0.5em]
      \item[] Iterate steps a--c until \(\beta_k\) minimizes the sample analogue of the moment condition
    \end{enumerate}
  \end{itemize}
\end{frame}

\begin{frame}{}
  Levinsohn, James and Amil Petrin (2003). Estimating Production Functions Using Inputs to Control for Unobservables. \textit{Review of Economic Studies 70}(2), 317-341.
  \begin{center}
    \includegraphics[height=0.4\textheight]{levinsohn.jpeg} \hspace*{1em}
    \includegraphics[height=0.4\textheight]{petrin.jpeg}
  \end{center}
\end{frame}
\addtocounter{framenumber}{-1}

\begin{frame}{}
Levinsohn and Petrin (2003 REStud). Estimating Production Functions Using Inputs \\ \phantom{you}
  \begin{center}
    \includegraphics[height=0.4\textheight]{levinsohn.jpeg} \hspace*{1em}
    \includegraphics[height=0.4\textheight]{petrin.jpeg}
  \end{center}
\end{frame}
\addtocounter{framenumber}{-1}

\begin{frame}{}
Levinsohn and Petrin (2003 REStud). Estimating Production Functions Using Inputs
  \begin{itemize}
    \vfill\item Investment is likely not monotonic in productivity
    \vfill\item If it were, firms with \(k_{jt}=k\) and \(i_{jt}=0\) (many in data!) had the same productivity
    \vfill\item {\color{MyStructure}\textbf{Research question:}} How to fix OP?
    \vfill\item {\color{MyStructure}\textit{Million dollar idea:}} use intermediate inputs instead
    \begin{align*}
      m_{jt} &= \tilde{m}_t (k_{jt}, \omega_{jt}) \quad \Rightarrow \omega_{jt} = \tilde{m}^{-1}_t(k_{jt}, m_{jt}) \\[0.5em]
      \Rightarrow va_{jt} &= \beta_0 + \beta_l l_{jt} + \underbrace{\beta_k k_{jt} + \tilde{m}^{-1}_t(k_{jt}, m_{jt})}_{\doteq \gamma_t (k_{jt}, m_{jt})} + \varepsilon_{jt}
    \end{align*}
    \vfill\item Estimation is analogous to OP
    \vfill\item Extra moment condition for the extra parameter: \(\E[(\xi_{jt} + \varepsilon_{jt}) m_{jt}] = 0\)
  \end{itemize}
\end{frame}

\begin{frame}{}
Ackerberg, Daniel A., Kevin Caves and Garth Frazer (2015). Identification Properties of Recent Production Function Estimators. \textit{Econometrica 83}(6), 2411-2451.
  \begin{center}
    \includegraphics[height=0.4\textheight]{ackerberg.jpeg} \hspace*{1em}
    \includegraphics[height=0.4\textheight]{caves.jpeg} \hspace*{1em}
    \includegraphics[height=0.4\textheight]{frazer.jpeg}
  \end{center}
\end{frame}
\addtocounter{framenumber}{-1}


\begin{frame}{}
Ackerberg, Caves and Frazer (2015 ECMA). Identification of Production Functions \\ \phantom{,ou}
  \begin{center}
    \includegraphics[height=0.4\textheight]{ackerberg.jpeg} \hspace*{1em}
    \includegraphics[height=0.4\textheight]{caves.jpeg} \hspace*{1em}
    \includegraphics[height=0.4\textheight]{frazer.jpeg}
  \end{center}
\end{frame}
\addtocounter{framenumber}{-1}

\begin{frame}{}
Ackerberg, Caves and Frazer (2015 ECMA). Identification of Production Functions
  \begin{itemize}
    \vfill\item Labor cannot be chosen independently of capital and intermediate inputs!
    \begin{align*}
      l_{jt} = \tilde{l}_t(k_{jt}, \omega_{jt}) = \tilde{l}_t(k_{jt}, \tilde{m}^{-1}_t(k_{jt}, m_{jt})) \doteq \tilde{\tilde{l}}_{t} (k_{jt}, m_{jt})
    \end{align*}
    \vfill\item {\color{MyStructure}\textbf{Research question:}} How to fix OP/LP?
    \vfill\item {\color{MyStructure}\textit{Million dollar idea:}} drop first stage and express \(m_{jt} = \tilde{m}_t (l_{jt}, k_{jt}, \omega_{jt})\)
    \begin{align*}
      va_{jt} &= \beta_0 + \beta_l l_{jt} + \beta_k k_{jt} + \omega_{jt} + \varepsilon_{jt} \\[0.5em]
      \Rightarrow va_{jt} &= \underbrace{\beta_0 + \beta_l l_{jt} + \beta_k k_{jt} + \tilde{m}^{-1}_t(l_{jt}, k_{jt}, m_{jt})}_{\doteq \gamma_t (l_{jt}, k_{jt}, m_{jt})} + \varepsilon_{jt}
    \end{align*}
    \vfill\item Estimation is analogous to 2nd stage of OP/LP
    \vfill\item Extra moment condition for the extra parameter: \(\E[\xi_{jt} l_{jt-1}] = 0\)
  \end{itemize}
\end{frame}

\begin{frame}{}
Gandhi, Amit, Salvador Navarro and David A. Rivers (2020). On the Identification of Gross Output Production Functions. \textit{Journal of Political Economy 128}(8), 2973-3016.
  \begin{center}
    \includegraphics[height=0.4\textheight]{gandhi.jpeg} \hspace*{1em}
    \includegraphics[height=0.4\textheight]{navarro.jpg} \hspace*{1em}
    \includegraphics[height=0.4\textheight]{rivers.jpg}
  \end{center}
\end{frame}
\addtocounter{framenumber}{-1}

\begin{frame}{}
Gandhi, Navarro and Rivers (2020 JPE). Gross Output Production Functions \\ \phantom{you}
  \begin{center}
    \includegraphics[height=0.4\textheight]{gandhi.jpeg} \hspace*{1em}
    \includegraphics[height=0.4\textheight]{navarro.jpg} \hspace*{1em}
    \includegraphics[height=0.4\textheight]{rivers.jpg}
  \end{center}
\end{frame}
\addtocounter{framenumber}{-1}

\begin{frame}{}
Gandhi, Navarro and Rivers (2020 JPE). Gross Output Production Functions 
  \begin{itemize}
    \vfill\item OP/LP/ACF estimate value added so \(m_{jt}\) does not directly enter production
    \vfill\item \ldots except if production is Leontief and firms perfectly optimize in each \(t\):
    \begin{align*}
      va_{jt} &= \min \lbrace f(l_{jt}, k_{jt}) + \omega_{jt}, g(m_{jt}) \rbrace + \varepsilon_{jt} \\
      \Rightarrow\text{gross output } go_{jt} &= \min \lbrace f(l_{jt}, k_{jt}) + \omega_{jt}, g(m_{jt}) \rbrace + m_{jt} + \varepsilon_{jt}
    \end{align*}
    \vfill\item This functional form is rather restrictive
    \vfill\item {\color{MyStructure}\textbf{Research question:}} How can we estimate \(go_{jt} = f(l_{jt}, k_{jt}, m_{jt}) + \omega_{jt} + \varepsilon_{jt}\)?
    % \vfill\item {\color{MyStructure}\textit{Million dollar idea:}} use the firm's first-order condition
  \end{itemize}
\end{frame}

\begin{frame}{}
Gandhi, Navarro and Rivers (2020 JPE). Gross Output Production Functions 
  \begin{itemize}
    \item {\color{MyStructure}\textbf{Research question:}} How can we estimate \(go_{jt} = f(l_{jt}, k_{jt}, m_{jt}) + \omega_{jt} + \varepsilon_{jt}\)?
    \vfill\item {\color{MyStructure}\textit{Million dollar idea:}} use the firm's first-order condition for identification
    \vfill\item With price-taking firms (\(P_t\), \(R_t\)) \& predetermined labor and capital:
    \begin{align*}
      \max_{M_{jt}} &\, P_t \E_t [F(L_{jt}, K_{jt}, M_{jt}) e^{\omega_{jt} + \varepsilon_{jt}}] - R_t M_{jt}
    \end{align*}
    \pause
    \begin{align*}
      \leadsto 0 =&\, P_t \frac{\partial}{\partial M_{jt}} F(L_{jt}, K_{jt}, M_{jt}) e^{\omega_{jt}} \E_t [e^{\varepsilon_{jt}}] - R_t \\
      f&(l_{jt}, k_{jt}, m_{jt}) + \omega_{jt} = r_t - p_t - \ln (\E_t [e^{\varepsilon_{jt}}]) \\
      \Rightarrow m_{jt} &= \tilde{m}_t (l_{jt}, k_{jt}, \omega_{jt}) \quad \Rightarrow \omega_{jt} = \tilde{m}^{-1}_t (l_{jt}, k_{jt}, m_{jt})
    \end{align*}
    \vfill\item Estimation is similar to ACF
  \end{itemize}
\end{frame}

\begin{frame}{Taking stock}
  \begin{itemize}
    \vfill\item Production function estimation is useful everywhere
    \begin{itemize}
      \vfill\item Academia: macro, labor, education, \ldots
      \vfill\item Industry: antitrust, litigation, Big Tech, \ldots
    \end{itemize}
    \vfill\item Canned estimation packages
    \begin{itemize}
      \vfill\item OP/LP/ACF: \texttt{prodest} in Stata
      \vfill\item GNR: \texttt{gnrprod} in R
    \end{itemize}
  \end{itemize}
\end{frame}

{
\setbeamercolor{background canvas}{bg=sidenotebg}
\begin{frame}[fragile]{Where to find data \myhfill{Sidenote}}
  \begin{itemize}
    \vfill\item Minimize GIGO (garbage in, garbage out)
    \vfill\item Search for as granular data as possible
    \vfill\item Always explore the raw data first!
    \vfill\item Useful micro data sources:
    \begin{itemize}
      \addtolength{\baselineskip}{0.5em}
      \item CERS Databank: go-to source for Hungarian micro data
      \item National Longitudinal Survey of Youth (NLSY97, NLSY79): US labor data by BLS
      \item Survey of Income and Program Participation (SIPP) by Census Bureau
      \item EU Statistics on Income and Living Conditions (EU-SILC) by Eurostat
      \item Augment with simulated tax data (TAXSIM for US, EUROMOD for EU, TAXBEN for UK)
      \item \href{https://docs.google.com/spreadsheets/d/1xjw5Z0LLK4Fp-vrEK523uvMnSfxtJOb_5f_vKUrZJDg/edit#gid=0}{\color{MyStructure}\underline{Collection of several data sources here}} (also saved in materials)
      \item Friends and family
    \end{itemize}
  \end{itemize}
\end{frame}

\begin{frame}{Unified approach in practice \myhfill{Sidenote}}
  \begin{enumerate}[a.]
    \vfill\item Find a broad topic that interests you
    \vfill\item Find the most granular base data available
    \vfill\item Identify the strengths of the data
    \begin{itemize}
      \item Unique information, policy variation, \ldots
    \end{itemize}
    \vfill\item 
    \begin{enumerate}[1.]
      \vfill\item[0.] Formulate a relevant and important research question
      \vfill\item Describe the underlying economic behavior
      \vfill\item Describe the ideal experiment that could identify the structural parameters
      \vfill\item Attempt to reduce the model
      \vfill\item Quantify policy implications
    \end{enumerate}
    \vfill\item[] Iterate until you have a strong pitch for a paper
  \end{enumerate}
\end{frame}
}

\begin{frame}{Frontiers of IO}
  \begin{itemize}
    \vfill\item {\color{MyStructure}\textbf{Central research question in IO:}} How do firms produce?
    \begin{itemize}
      \item Production function estimation: we got to the frontier
      \item \ldots
    \end{itemize}
    \vfill\item Other central questions:
    \begin{itemize}
      \item How do firms respond to demand shocks?
      \item What is the origin and dynamics of product market power?
      \item How do firms make entry/exit decisions?
    \end{itemize}
    \vfill\item Things we do not cover:
    \begin{itemize}
      \item Industry dynamics
      \item Strategic interactions (empirical games)
      \item \ldots
      \item See Sergey Lychagin's syllabus for more
    \end{itemize}
  \end{itemize}
\end{frame}



\end{document}

